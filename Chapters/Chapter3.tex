% Chapter 3
\chapter{Government Sector Disaggregation in the 2009 Scottish SAM}
\label{Chapter3}

%%%%%%%%%%%%%%%%%%%%%%%%%%%%%%%%%%%%%%%%%%%%%%%%%%%%%%%%%%%%%%%%%%%%%%%%%%%%%%%%%%%%%%%%%%%%%%%%%%%%%%%%%%%%%%%%%%%%%%%%
%%%       SECTION 1
%%%%%%%%%%%%%%%%%%%%%%%%%%%%%%%%%%%%%%%%%%%%%%%%%%%%%%%%%%%%%%%%%%%%%%%%%%%%%%%%%%%%%%%%%%%%%%%%%%%%%%%%%%%%%%%%%%%%%%%%


\section{Introduction}
\label{sec:3.1}

This chapter shows how the the Government Sector in the 2009 Scottish SAM is disaggregated into three separate government accounts. These accounts are the UK Government in Scotland, the Scottish Government and the Local Government in Scotland, i.e. Local Authorities. This work extends both the Income and Expenditure (IncExp) Accounts and the aggregated SAM presented in Chapter \ref{Chapter2}. 

%%%%     MOTIVATION    %%%%%%

%   Motivation here. Why do we want to disaggregate it, what are the desired outcomes and what can we do with it. 
% In order to allow for a more detailed analysis of the role of the government sector on the Scottish economy, this sector needs to be disaggregated. Three separate public sectors are identified with the UK Government in Scotland (UK Gov in Scotland), the Scottish Government (Scot Gov) and the Local Government in Scotland (Local Gov). 
% A disaggregated government sector in the SAM allows for inter alia a more detailed analysis of governmental transfer payments. Furthermore, the disaggregation enables more precise modelling of the impact of, for example, tax policies on the Scottish economy. 
%%%%

Section \ref{sec:3.2} discusses the Government Sector in the aggregated SAM in more detail and highlights how the Government Sector is treated in the IO Tables and in the SAM, respectively. Section \ref{sec:3.3} details how the IncExp Accounts are extended to allow for the disaggregation of the three government accounts outlined above. The disaggregated Government Sector in the SAM is presented in Section \ref{sec:3.4}. Lastly, Section \ref{sec:3.5} outlines the method for the disaggregation of cells in the SAM originating from the IO Tables and the IncExp Accounts, respectively.

\newpage

%%%%%%%%%%%%%%%%%%%%%%%%%%%%%%%%%%%%%%%%%%%%%%%%%%%%%%%%%%%%%%%%%%%%%%%%%%%%%%%%%%%%%%%%%%%%%%%%%%%%%%%%%%%%%%%%%%%%%%%%
%%%       SECTION 2
%%%%%%%%%%%%%%%%%%%%%%%%%%%%%%%%%%%%%%%%%%%%%%%%%%%%%%%%%%%%%%%%%%%%%%%%%%%%%%%%%%%%%%%%%%%%%%%%%%%%%%%%%%%%%%%%%%%%%%%%


\section{The Government Sector in the IO Tables}
\label{sec:3.2}

The government sector in the 2009 Scottish IxI Table (IO Tables) is split between Central Government and Local Government. The former comprises of both the UK Gov in Scotland and the Scot Gov whilst the latter is the Local Gov as described above. Table \ref{tab:3.2.1} shows an aggregated version of the 2009 Scottish IO Tables similar to Table \ref{tab:2.3.1}. In contrast to Table \ref{tab:2.3.1}, Table \ref{tab:3.2.1} shows two government sectors as presented in the IO Tables and the External sector is aggregated into one. The IO Tables give a breakdown of the ``Activities'' of the Central Government at \textsterling19,296m and of the Local Government at  \textsterling10,190m. These are the sums of the expenditure of the two government sectors on products sold in the Scottish economy. With regards to the revenue earned by the government sector in Scotland, the IO Tables do not provide any further disaggregated figures. The total receipts of the government sector are attributed to the Central Government at \textsterling13,165m, which are tax payments.

%The IO Tables outlined here are embedded in the SAM. They are the biggest data source and the framework of the SAM is an extension of the IxI Tables. 

\bigskip

  \begin{table}[H] \caption{Aggregate Industry by Industry Table, 2009 basic prices (\textsterling million)}
  \bigskip \begin{scriptsize} \begin{centering} \begin{doublespacing}
  \begin{tabular}{lrrrrrrrrr}
  \toprule
  & \begin{sideways}Activities \end{sideways} &
  \begin{sideways}Labour \end{sideways} &
  \begin{sideways}Capital \end{sideways} &
  \begin{sideways}Other Value Added \end{sideways} &
  \begin{sideways}Households \end{sideways} &
  \begin{sideways}Central Government \end{sideways} &
  \begin{sideways}Local Government \end{sideways} &
  \begin{sideways}External  \end{sideways} &
  \begin{sideways}Total \end{sideways}  \bigstrut[b]\\
  \hline
  1. Activities &  63,607  & \: \: \: \: \: -  &  13,981  &  \: \: \: \: \: -  &  49,802  &  19,296  &  10,190  &  54,045  &  210,920  \\
  2. Labour &  63,561  &  -  &  -  &  -  &  -  &  -  &  -  &  -  &  63,561  \\
  3. Capital &  -  &  -  &  -  &  -  &  -  &  -  &  -  &  -  &  -  \\
  4. Other Value Added &  38,441  &  -  &  -  &  -  &  -  &  -  &  -  &  -  &  38,441  \\
  5. Households &  -  &  -  &  -  &  -  &  -  &  -  &  -  &  -  &  -  \\
  6. Central Government &    4,779  &  -  &  1,495  &  -  &  6,568  &  -  &   -  &      322  &  13,165  \\
  7. Local Government &  -  &  -  &  -  &  -  &  -  &  -  &  -  &  -  &  -  \\
  8. External &  40,532  &  -  &  4,455  &  -  &  18,299  &  -  &  7,419 & 3,051  &  73,755   \bigstrut[b]\\
  \hline
  Total &  210,920  &  -  &  19,930  &  -  &  74,669  &  19,296  &  10,190  &  64,837  &   \bigstrut[t]\\
      \bottomrule \end{tabular}%
      \bigskip \begin{flushright}Source: \citeA{ScottishGovernment2013a} \end{flushright} \label{tab:3.2.1}
      \end{doublespacing} \end{centering} \end{scriptsize} \end{table}

\bigskip
\newpage 


%%%%%%%%%%%%%%%%%%%%%%%%%%%%%%%%%%%%%%%%%%%%%%%%%%%%%%%%%%%%%%%%%%%%%%%%%%%%%%%%%%%%%%%%%%%%%%%%%%%%%%%%%%%%%%%%%%%%%%%%
%%%       SECTION 3
%%%%%%%%%%%%%%%%%%%%%%%%%%%%%%%%%%%%%%%%%%%%%%%%%%%%%%%%%%%%%%%%%%%%%%%%%%%%%%%%%%%%%%%%%%%%%%%%%%%%%%%%%%%%%%%%%%%%%%%%

\section{The Government Sector in the IncExp Accounts}
\label{sec:3.3}

This Section shows how the Government Sector is disaggregated in the IncExp Accounts. The 2009 Scottish SAM was initially constructed with one government sector (see Chapter \ref{Chapter2}). This of public sector aggregation has been employed by the majority of SAMs for Scotland (REF? and also UK SAMs??). Therefore, the government sector was unified in the 2009 Scottish IncExp Accounts. Accordingly, any tax payments or transfer payments to or from the ``Government'' were identified in a single public sector account. Section \ref{sec3.1} presented the motivation for why the aggregated treatment of the government sector, however, is insufficient.

\bigskip

This Section is structured as follows. First, Section \ref{sec:3.3.1} outlines the extended IncExp Accounts, including all three government sectors identified above. Here, the flow of funds between the government accounts and the other accounts in the IncExp Accounts are analysed. Next, the extended ``Data'' section (see Section 2.??) is outlined. Further, the ``Shares'' dataset is analysed as well as any additional calculations needed to produce the disaggregated IncExp Accounts.

\bigskip

\subsection{Disaggregating the Government Sector in the IncExp Accounts}
\label{sec:3.3.1}

\subsubsection{Layout}

Disaggregating the Government Sector in the SAM requires extended IncExp Accounts. First, two new government accounts are added to the IncExp framework (see ??), resulting in three government accounts in total. Second, the linkages between the Household, Corporate and Capital account to the Government. This allows for payments to and transfers from the Local, Scottish and UK Government in Scotland to the other accounts above.

\bigskip

The three Government Accounts here are structured in the same way as the aggregated Government Account used for the 2009 Scottish SAM (see ?? and xx from ch 2).  This set-up enables provides familiarity for users of the aggregated Accounts as well as allow for further disaggregation of the Government Account. For example, the inter-governmental transfers are not modelled here explicitly due to data availability, but this could be incorporated given the framework here.

\bigskip

\subsubsection{Data}

The data for the 2009 Scottish SAM disaggregated by Government is mainly the same as used in the aggregated SAM. Thus the discussion on data sources for the 2009 Scottish SAM applies here, see Ch 2 Section xx. 

\bigskip

For the disaggregation of the Government sector, the Central and Local Government split in the IxI Tables are used. This separation allows identification of the flow of funds to the Local Government, i.e. Local Authorities in Scotland. However, this split does not provide disaggregation of Scottish Government versus UK Government in Scotland spending/ receipts. Thus further data is needed in order to arrive at estimates for the two components of the ``Central Government'' in Scotland. 

\bigskip

The sources are:
- GERS
- Scottish Local Gov Financial Statistics

They are used how....

In the Calculation Section below: say how entries are calculated directly.

Where this is not possible, use shares. 


\bigskip

\subsubsection{Calculation}

Only direct entries here.
-HH income from gov (benefits) - only local and uk
-HH payments to gov (tax) - only local and uk



\bigskip

\subsubsection{Shares}

The largest part of the disaggregated Government sector is computed using ``Shares''. In particular, all governmental expenditure by industry is derived using shares. This is approach ensures consistency with the totals derived in the 2009 Scottish SAM (REF).

- Gov OVA Income
- Gov Cap Expenditure

\bigskip


%%%%%%%%%%%%%%%%%%%%%%%%%%%%%%%%%%%%%%%%%%%%%%%%%%%%%%%%%%%%%%%%%%%%%%%%%%%%%%%%%%%%%%%%%%%%%%%%%%%%%%%%%%%%%%%%%%%%%%%%
%%%       SECTION 4
%%%%%%%%%%%%%%%%%%%%%%%%%%%%%%%%%%%%%%%%%%%%%%%%%%%%%%%%%%%%%%%%%%%%%%%%%%%%%%%%%%%%%%%%%%%%%%%%%%%%%%%%%%%%%%%%%%%%%%%%

\section{The Disaggregated Government Sector in the 2009 Scottish SAM}
\label{sec:3.4}

Table \ref{tab:3.5.1} shows an aggregated version of the 2009 Scottish SAM with a fully disaggregated government sector.

%%%%%%%%%%%%%%%%%%%%%%%%%%%%%%%%%%%%%%%%%%%%%%%%%%%%%%%%%%%%%%%%%%%%%%%%%%%%%%%%%%%%%%%%%%%%%%%%%%%%%%%%%%%%%%%%%%%%%%%%
This section describes what the new SAM looks like. Further it highlights in detail the figures, which are now disaggregated and should also explain what we see here, i.e. that many taxes to this part of the public sector, etc.
%%%%%%%%%%%%%%%%%%%%%%%%%%%%%%%%%%%%%%%%%%%%%%%%%%%%%%%%%%%%%%%%%%%%%%%%%%%%%%%%%%%%%%%%%%%%%%%%%%%%%%%%%%%%%%%%%%%%%%%%

\bigskip
\begin{sidewaystable}[htbp] \caption{Aggregated 2009 SAM for Scotland, 2009 basic prices (\textsterling million)}
  \bigskip \begin{scriptsize} \begin{centering} \begin{doublespacing}
  \begin{tabular}{lrrrrrrrrrrr}
  \toprule
  & \begin{sideways}1. Activities \end{sideways} &
  \begin{sideways}2. Labour \end{sideways} &
  \begin{sideways}3. Capital \end{sideways} &
  \begin{sideways}4. Other Value Added \end{sideways} &
  \begin{sideways}5. Households \end{sideways} &
  \begin{sideways}6. Corporations \end{sideways} &
  \begin{sideways}7. UK Government in Scotland \end{sideways} &
  \begin{sideways}8. Scottish Government \end{sideways} &
  \begin{sideways}9. Local Government \end{sideways} &
  \begin{sideways}10. External \end{sideways} &
  \begin{sideways} Total \end{sideways} \bigstrut[b]\\
  \hline
  1. Activities  & 63,607   & - & 13,981 & - & 49,802   & - & 19,296 &
  - & 10,190 & 54,045 &  210,920  \bigstrut[t]\\
  2. Labour    & 63,561   & - & - & - & -    & - & - & - & - & - & 63,561  \\
  3. Capital   & -    & - & - & - &   5,070  & 24,828 & 119  & -  & -
   & - 10,087 & 19,930  \\
  4. Other Value Added    & 38,441   & - & - & - & -  & -  & - & - & - & - &
  38,441  \\
  5. Households  & -    & 63,561 & - & 5,289 & -    & 15,103 & 19,835 &
  - & - & 4,090 & 107,877  \\
  6. Corporations & -    & - & - & 29,456 &   6,401  & - & 5,722 & - & - &
  11,928 & 53,507  \\
  7. UK Government in Scotland  &   4,779  & - & 1,495 & 3,697 & 27,947   & 5,248 &
  13,165 & - &  - & 20,363  & 76,694  \\
  8. Scottish Government  & - & - & - & - & -  & -  & - & - & - & - & -  \\
  9. Local Government  & - & - & - & - & -  & -  & - & - & - & - & -  \\
  10. External  & 40,532   & - & 4,455 & - &  18,657  & 8,328 & 8,368 & - & -
  & 10,470  \bigstrut[b]\\
  \hline
  Total   & 210,920 & 63,561 & 19,930 & 38,441 & 107,877 & 53,507 & 76,694
  & - & - & 90,808 &  \bigstrut[t]\\
    \bottomrule \end{tabular}%
    \bigskip \begin{center}The disaggregated SAM can be found at:
    \url{https://www.strath.ac.uk/fraser/research/XXXX}\end{center} \label{tab:3.5.1}
    \end{doublespacing} \end{centering} \end{scriptsize} \end{sidewaystable}

\newpage

%%%%%%%%%%%%%%%%%%%%%%%%%%%%%%%%%%%%%%%%%%%%%%%%%%%%%%%%%%%%%%%%%%%%%%%%%%%%%%%%%%%%%%%%%%%%%%%%%%%%%%%%%%%%%%%%%%%%%%%%
%%%       SECTION 5
%%%%%%%%%%%%%%%%%%%%%%%%%%%%%%%%%%%%%%%%%%%%%%%%%%%%%%%%%%%%%%%%%%%%%%%%%%%%%%%%%%%%%%%%%%%%%%%%%%%%%%%%%%%%%%%%%%%%%%%%

\section{The Disaggregated Government Sector in the IncExp Accounts - Method}
\label{sec:3.5}

\bigskip

