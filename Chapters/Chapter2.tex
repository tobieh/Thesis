% Chapter 2
\chapter{Social Accounting Matrix for Scotland}
\label{Chapter2}

\section{Introduction} 
\label{sec:2.1}

This chapter outlines the methodology and computations used to construct the 2009 Social Accounting Matrix (SAM) for Scotland. A SAM can be described as a static image (a snapshot) of the flow of goods, services and factors, and the concurrent flow of funds between agents in an economic system for a given time-period \shortcite{Hosoe2010a}. Essentially, the SAM extends the Scottish Input-Output (IO) tables by incorporating Income and Expenditure (IncExp) Accounts. Thus, the IncExp Accounts contain information on institutional accounts that are not recorded within the IO tables. Therefore the SAM can be used to analyse social and economic policy in a more comprehensive way. The main benefits and the structure of a SAM are outlined in the first sections.  Next, the computed IncExp Accounts and the 2009 Scottish IO tables are combined to complete the 2009 SAM for Scotland. In the last sections, the methodology required to compute the IncExp Accounts is described in detail.

%%%%%%%%%%%%%%%%%%%%%%%%%%%%%%%%%%%%%%%%%%%%%%%%%%%%%%%%%%
%SECTION
%%%%%%%%%%%%%%%%%%%%%%%%%%%%%%%%%%%%%%%%%%%%%%%%%%%%%%%%%%
\newpage
\section{Social Accounting Matrices} 
\label{sec:2.2}

The SAM can be considered as an extension to an IO table which not only records macroeconomic-aggregates but also the distribution and redistribution of income. The focus of a SAM therefore lies in recording interrelationships at the meso-level with emphasis on distributive aspects \cite{Keuning1988a}. A SAM can therefore be described as being concerned with the systematic organisation of information about the economic and social structure of a country, region, city or other unit, in a particular time period - usually a year \cite{King1981a}.

\bigskip

In contrast to IO tables, the SAM records flows from producing sectors to factors of production and then on to institutional accounts and finally back to the demand for goods \cite{Adelman1986a}. As such, a SAM is different from an IO table in that it contains complete information on institutional accounts (i.e. households, government and corporations), instead of solely tracing income and expenditure flows of activities and commodities \shortcite{Breisinger2010a}. The main features of a SAM can be divided into three sections \cite{Round2003a}.   

\bigskip

First, the row sums in the SAM show the total receipts and the column sums show the total payments of funds. Importantly, each row sum must equal its corresponding column sum. That is, the total revenue must equal total expenditure in each account \shortcite{Hosoe2010a}. Each cell in the SAM represents a flow of funds from a column account to a row account, thereby documenting the interconnections between these accounts in an explicit way and identifying the source and use of all transactions. 

\bigskip

Second, the SAM is considered to be comprehensive as it shows economic activity in terms of consumption, production, accumulation and distribution (although not necessarily in equivalent detail). 

\bigskip

Third, the SAM is considered to be flexible in the degree of disaggregation, whilst at the same time following a basic accounting framework \shortcite{Breisinger2010a}. The degree of disaggregation generally depends on the motivation behind constructing the SAM (e.g. depending on the location of the initial shock and the outcome variables) and more restrictively, the availability of data \cite{Round2003a}. 

\bigskip

The benefits arising from computing a SAM are multifold. The additional information contained in the SAM, compared to IO tables, can be used to extend and improve the multiplier modelling capacity to include the behaviour of the non-production part of the economy. In particular, the link between activity and changes in household income should improve the Type II multiplier. 

\newpage

Moreover, the SAM can incorporate a highly disaggregated social breakdown. This is particularly important as a large number of economic interactions happen within the household sector. That is, income from labour and the household sector can be further broken down to analyse distributional effects of policy more accurately \cite{Stuttard2003b}. 

\bigskip

An important side-effect of the compilation process of a SAM is that data gaps and inconsistencies can be identified. This information can be used to improve and extend survey methodologies, definitions and classifications and overall compatibility of data sources \cite{Keuning1988a}. 

\bigskip

The main utility, however, of a SAM is that it provides a comprehensive and consistent record of the interrelationships of an economy at the level of individual production sectors, factors, and institutions. Thereby, the SAM makes available an internally consistent statistical foundation, or benchmark, for the creation of plausible economic models (e.g. Computable General Equilibrium models) which simulate changes to the economy \cite{Reinert1997a}.     

%%%%%%%%%%%%%%%%%%%%%%%%%%%%%%%%%%%%%%%%%%%%%%%%%%%%%%%%%%
%SECTION
%%%%%%%%%%%%%%%%%%%%%%%%%%%%%%%%%%%%%%%%%%%%%%%%%%%%%%%%%%
\newpage
\section{Social Accounting Matrix for Scotland} 
\label{sec:2.3}

The main components of the Scottish SAM are the latest IO tables for Scotland \cite{ScottishGovernment2013a} and the IncExp Accounts. More precisely, the 2009 Industry by Industry (IxI) basic-price IO table for Scotland is used. This is a symmetric IxI IO table with 104 industries defined using the SIC07 classification. The IxI table records the destination of industry outputs. The data on industry linkages can be used to analyse knock-on effects throughout the Scottish economy of a change of final demand \cite{ScottishGovernment2011a}. 

\bigskip

Table \ref{tab:2.3.1} depicts an aggregate version of the SAM that is derived by combining the IxI table and the IncExp Accounts. For illustration disaggregation within accounts has been suppressed. For example, the 104 industries contained in the SAM are aggregated to one industry (Activities). However, it must be emphasised that for modelling purposes a more detailed SAM is used. The aggregated 2009 SAM for Scotland is a square matrix with 8 column and 8 row accounts. This aggregated account has: Activities, Factors (labour + capital), Institutions (Households, Corporations and Government), External Account, and Other Value Added (OVA). The SAM treats both rows and columns as accounts. The row and column entries derived from the IO tables are considered to be sales and expenditures receptively. In contrast, the accounts derived within the IncExp Accounts are considered to be transfers. 

\bigskip

\begin{table}[H] \caption{Aggregated 2009 SAM for Scotland, 2009 basic prices (\textsterling million)}
\bigskip \begin{scriptsize} \begin{centering} \begin{doublespacing}
    \begin{tabular}{lrrrrrrrrr}
          \toprule
          & \begin{sideways}1. Activities (IOC1-104)\end{sideways} & \begin{sideways}2. Households\end{sideways} & \begin{sideways}3. Corporate\end{sideways} & \begin{sideways}4. Government\end{sideways} & \begin{sideways}5. Capital\end{sideways} & \begin{sideways}6. Employment Income\end{sideways} & \begin{sideways}7. Exports to RUK + ROW\end{sideways} & \begin{sideways}8. Other Value Added\end{sideways} & \begin{sideways}Total (Receipts)\end{sideways} \bigstrut\\
    \hline
    1. Activities (IOC1-104) & 63,607 & 49,802 & -     & 29,486 & 13,981 & -     & 54,045 & -     & 210,920 \bigstrut[t]\\
    2. Households & -     & -     & 15,104 & 19,835 & -     & 63,561 & 4,088 & 5,289 & 107,877 \\
    3. Corporate & -     & 6,401 & -     & 5,722 & -     & -     & 11,928 & 29,456 & 53,507 \\
    4. Government & 4,779 & 27,947 & 5,248 & 13,165 & 1,495 & -     & 20,363 & 3,697 & 76,694 \bigstrut[b]\\
    5. Capital & -     & 5,070 & 24,826 & 119   & -     & -     & -10,086 & -     & 19,930 \bigstrut[t]\\
    6. Employment Income & 63,561 & -     & -     & -     & -     & -     & -     & -     & 63,561 \\
    7. Imports from RUK + ROW & 40,532 & 18,657 & 8,328 & 8,368 & 4,455 & -     & 10,470 & -     & 90,808 \\
    8. Other Value Added & 38,441 & -     & -     & -     & -     & -     & -     & -     & 38,441 \bigstrut[b]\\
    \hline
    Total (Expenditures) & 210,920 & 107,877 & 53,507 & 76,694 & 19,930 & 63,561 & 90,808 & 38,441 &  \bigstrut\\
\bottomrule 
\end{tabular}%  
\bigskip \begin{flushright} \end{flushright} \label{tab:2.3.1} 
\end{doublespacing} \end{centering} \end{scriptsize} \end{table} \bigskip

\newpage

The first row of the SAM, for example, can be read as follows: raw material purchases of goods within Scotland (\textsterling63,607m), Household consumption expenditure on goods$/$services (\textsterling49,802m), Government current expenditure on Activities (\textsterling29,486m), investment expenditure on Scottish goods (\textsterling13,981m), exports to RUK + ROW (\textsterling54,045m) and the total of (\textsterling210,920m) represents total aggregate demand of gross outputs. Conversely, the first column can be read as: raw material sales of goods within Scotland (\textsterling63,607m), Government current income from Activities (\textsterling4,779m), Employment income from Activities (\textsterling63,561m), imports to RUK + ROW (\textsterling40,532m), Other Value Added income from Activities (\textsterling38,441m), and the total of (\textsterling210,920m) represents total aggregate supply of gross outputs.   

\bigskip 

Due to extending the IxI table by the IncExp Accounts, the expenditures for all accounts are equal to their receipts (this is not possible without the IncExp Accounts). Hence, each account in the Scottish SAM is balanced by its corresponding account. For example, Government expenditures (\textsterling115,136m) are balanced by Government receipts (\textsterling115,136m). That is, constructing the SAM by extending IO tables by the IncExp Accounts does not require any rebalancing. The IO table is fully incorporated without the need of changing any entries thereof, i.e. the entries within the first row and column in the SAM stem solely from the IO accounts. All cells that were added to the IO table to compute the SAM are balanced within the IncExp Accounts so that total revenue equal total expenditure in each account. This approach assures that the integrity of the IO accounts is retained when constructing the SAM.

\bigskip

It must be emphasised again that the SAM is meant to fit around the existing IO tables and other national statistics. Data necessary for the construction of the SAM that are not contained within the IO table are derived by computing the IncExp Accounts. These accounts record income and expenditure of households, corporations, government, capital and the external sector in detail. The construction of the IncExp Accounts is outlined in the following section.

%%%%%%%%%%%%%%%%%%%%%%%%%%%%%%%%%%%%%%%%%%%%%%%%%%%%%%%%%%
%SECTION
%%%%%%%%%%%%%%%%%%%%%%%%%%%%%%%%%%%%%%%%%%%%%%%%%%%%%%%%%%
\newpage
\section{The Income and Expenditure Accounts for 2009}
\label{sec:2.4}

The IncExp Accounts provide detailed flows of funds for the main local transactors (Households, Corporations and Government), as well as for the Capital and External sectors in Scotland. The IncExp Accounts are compiled by using publicly available data, sourced from both the UK and the Scottish Government, including the 2009 IO Tables for Scotland. The above section outlined the role that the IncExp Accounts have in extending the IO Tables into a SAM for Scotland. This section provides an overview of the IncExp Accounts and how these Accounts are constructed. This includes an illustration of the layout of the Accounts, an overview of the calculations and the internal balancing and a discussion of the data sources. The following section provides a methodology guide on the computation of each entry in the Accounts.

\bigskip

\subsection{Layout}
\label{sec:2.4.1}

\bigskip

The IncExp Accounts (see Table \ref{tab:2.4.1}) are divided into five sectors (Households, Corporations, Government, Capital and External) and the Scottish Trade and External Balance with both the RUK and the ROW. Each of those sectors is divided further into an Income and an Expenditure section (left-hand side and right-hand side respectively), hence the name for these Accounts.

\bigskip

Each numerical entry in the IncExp Accounts (see Table \ref{tab:2.4.1}) is henceforth referred to as a cell and can be identified in two ways. First, each cell can be identified through a directory, e.g. Corporations > Income > Profit Income (OVA) or through the number code given to each entry, which is (Cell 19) for this example. The latter method is used when cross-referencing in the detailed breakdown of the cells in the Methodology of the IncExp Accounts (see Section \ref{sec:2.5}). Every sector has a Total Income and a Total Expenditure figure, which is a summation of the entries in each section (highlighted in bold). The Household and Government sector have additional Control Totals (the bottom totals in bold) from external sources, for example (Cell 8) and (Cell 17) for the Household Sector.

\bigskip

The Primary Sectors (Household, Corporations and Government) have similar cell breakdowns with Income/Payments to the other Primary Sectors as well as External Transfer payments comprising the largest share of entries. Additionally, all Primary Sectors have a Profit Income (OVA) entry and a Payments to Capital entry on the Income and on the Expenditure side, respectively. Cells with one star following the numerical entry refer to Balancing Items and those with two stars refer to Corresponding Figures. A detailed discussion of those entries can be found in ``Calculation Overview and Internal Balancing'' (see Section \ref{sec:2.4.2}).


% % % % % % % % % % % % % % % % % % % % % % % % % % % % % % % % %
% % % % % % THE INCOME AND EXPENDITURE ACCOUNTS TABLE % % % % % %
% % % % % % % % % % % % % % % % % % % % % % % % % % % % % % % % %


\begin{table}[H] \caption{Income-Expenditure Accounts for Scotland (in \textsterling million)}
\bigskip \begin{scriptsize} \begin{centering} \begin{spacing}{1.2}
    \begin{tabular}{lrllrl}
          \toprule
    \textbf{HOUSEHOLDS} \bigstrut\\
   \hline  
\bigstrut[t]    1. \textbf{Income} & \textbf{107877} & & 10. \textbf{Expenditure} & \textbf{107877} & \\
    2. Income from Employment & 63561  &   & 11. IO Expenditure & 74669 & \\
    3. Profit Income (OVA) & 5289 & & 12. Payments to Corporations & 6401 &* \\
    4. Income from Corporations & 15104 & & 13. Payments to Government & 21379 & \\
    5. Income from Government & 19835 & & 14. Transfers to ROW & 119 & \\
    6. Transfers from RUK & 1852 & & 15. Transfers to RUK & 238 & \\
    7. Transfers from ROW & 2237 & & 16. Payments to Capital (Savings) & 5070 &\\
    8. \textbf{Total Household Income} & \textbf{107877} & & 17. \textbf{Total Expenditure} & \textbf{107877} & \\
    9. \textbf{Mixed and Prop Income Unalloc.} & 0 & \bigstrut[b]\\
    \hline
\bigstrut[t]    \textbf{CORPORATIONS} \bigstrut\\
    \hline
\bigstrut[t] 18. \textbf{Income} & \textbf{53507}  &   & 24. \textbf{Expenditure} & \textbf{53507} & \\
       19. Profit Income (OVA) & 29456 & & 25. Payments to Households & 15104 &** \\
       20. Income from Households & 6401 &** & 26. Payments to Government & 5248 & \\
       21. Income from Government & 5722 &** & 27. Transfers to RUK & 3768 & \\
       22. Income from RUK & 5964 & & 28. Transfers to ROW & 4560 & \\
       23. Income from ROW & 5964 & & 29. Payments to Capital (Savings) & 24826 &* \\
    \hline
    \textbf{GOVERNMENT} \bigstrut\\
    \hline
\bigstrut[t]       30. \textbf{Income} & \textbf{63530}  &   & 37. \textbf{Expenditure} & \textbf{63530} & \\
       31. Profit Income (OVA) & 3697 & & 38. IO Expenditure & 29486 & * \\
       32. Net Commodity Taxes & 13165 & & 39. Payments to Corporations & 5722 &* \\
       33. Income from Households & 21379 &** & 40. Payments to Households & 19835 &** \\
       34. Income from Corporations & 5248 &** & 41. Transfers to RUK & 8368 & \\
       35. Income from RUK & 20041 &* & 42. Payments to Capital (Savings) & 119 & \\
       36. \textbf{Total Govt Inc Balancing Total} & \textbf{63530} &** & 43. \textbf{Total Govt Exp Balancing Total} & \textbf{63530} &  \\ 
    \hline
    \textbf{CAPITAL} \bigstrut\\
    \hline
\bigstrut[t]       44. \textbf{Income} & \textbf{19930}  &   & 49. \textbf{Expenditure} & \textbf{19930} & \\
       45. Households & 5070 &** & 50. IO Expenditure & 19930 &  \\
       46. Corporations & 24826 &** &  \\
       47. Government & 119 &** &  \\
       48. RUK/ROW & -10086 &** &  \\ 
    \hline
    \textbf{EXTERNAL} \bigstrut\\
    \hline
\bigstrut[t]  51. RUK Income from Scotland & 67133  &   & 58. RUK Expenditure in Scotland & 70595 & \\
       52. Goods \& Services & 54759 & & 59. Goods \& Services & 42739 & \\
       53. Transfers & 12374 & & 60. Transfers & 27857 & \\
       54. RUK Income from Scotland & 23676 & & 61. ROW Expenditure in Scotland & 27378 & \\
       55. Goods \& Services & 18997 & & 62. Goods \& Services & 19178 & \\
       56. Transfers & 4679 & & 63. Transfers & 8201 & \\
       & & & 64. Tourist Expenditure in Scotland & 2921 & \\
       57. \textbf{Total Income} & \textbf{90808} & & 65. \textbf{Total Expenditure} & \textbf{100894} &  \\  
        & & & 66. Surplus/Deficit &-10086 & \\   
    \hline
    \textbf{G\&S TRADE BALANCE} \bigstrut\\
    \hline
\bigstrut[t] \quad \enspace \thinspace Scotland with RUK and ROW & &   & \quad \enspace \thinspace Total Balance of Payments & & \\
       67. RUK & -12020 & & 69. RUK & 5215 &  \\
       68. ROW & 181 & & 70. ROW & 4871 & \\
       & & & 71. Total Balance of Payments & 10086 & \\  
    \hline    
    \textbf{EXTERNAL BALANCE} \bigstrut\\
    \hline
\bigstrut[t]   72. Income from Employment & -3462 & & & & \\
       73. Profit Income (OVA) & -3703 & & & &  \\
       74. Income from Corporations & -2921 & & & & \\
       75. Income from Government & -10086 & & & & \\ 
    \bottomrule  \bigstrut[b]\\
\qquad \qquad \qquad Balancing Item: *   & & & Row Entries (Element determines Column)  \\ 
\qquad \qquad \qquad Corresponding Figure: ** & & & Row Entries (Element determines Column)  \bigstrut\\       
   \end{tabular}%  
\bigskip \begin{flushright}\end{flushright} \label{tab:2.4.1}
\end{spacing} \end{centering}  \end{scriptsize} \end{table}

%%%%%%%%%%%%%%%%%%%%%%%%%%%%%%%%%%%%%%%%%%%%%%%%%%%%%%%%%%
%SECTION
%%%%%%%%%%%%%%%%%%%%%%%%%%%%%%%%%%%%%%%%%%%%%%%%%%%%%%%%%%
\subsection{Calculation Overview and Internal Balancing}
\label{sec:2.4.2}

\bigskip

The structure used for compiling the IncExp Accounts follows a framework set out by \shortciteA{Hermannsson2010a}. The data used for the calculation of the IncExp Accounts is presented in two formats. First, data is given for a calendar year, for example the GERS and ONS Blue Book figures \cite{ScotGov2013b,ONS2011c}. Secondly, data is presented in the UK financial year format, which spans, for instance, 01.04.2008 to 31.03.2009. In order to transform this data to the calendar year format 2009, which is the format used for the 2009 Scottish SAM, a one-quarter share of 2008-09 data is taken and a three-quarter share of 2009-10 data.

\bigskip

The majority of cells in the IncExp Accounts are derived from several data entries taken from a single source, for example, the annualised tax payments from households to Government (Cell 13). These figures are all taken from the Government Expenditure and Revenue Scotland (GERS) publication. 

\bigskip

Other cells, are derived from several data entries taken from a variety of data sources. For example, (Cell 19) is a combination of an entry taken straight from the 2009 IO Tables for Scotland as well as from (Cell 3) and (Cell 31), which in turn are calculated using ONS and Scottish Government data, respectively.

\bigskip

Additionally to the above-outlined data sources, the IncExp Accounts contain internally derived cells. These are notated with a single star (*) for Balancing Items and with two stars (**) for Corresponding Figures (see Table \ref{tab:2.4.1}). 

\bigskip

Balancing Items are used to, first, balance the Total Income and the Total Expenditure of the Primary Sectors. These would not balance without some manual adjustments, due to a variety of sources being used to calculate each sector within the IncExp Accounts. Secondly, data availability or quality is least robust for these cells and thus chosen to be Balancing Items. These entries are furthermore necessary to ensure the internal consistency of the IncExp Accounts, which in turn balances the SAM. Balancing Items are derived by summing up all figures of one sector on the relevant account side apart  and deducting the Total figure by that calculated sum. For example, (Cell29) is calculated through deducting the sum of (Cell 25) to (Cell 28) from the total Expenditure (Cell 24).

\bigskip

Corresponding Figures are cells which equal another entry within the IncExp Accounts. These entries are based on the assumption of the Circular Flow of the Economy, where, for example, the income that one sector receives from another is equal to the payment that the latter sector makes to the former. For example, the Income from Corporations received by the Government (Cell 34) is equal to the Payments to Government made by Corporations (cell 26). Further, all Income entries for the Capital Accounts are Corresponding Figures, as these are equal to the Payments to Capital entries by each of the Primary Sectors (Cells 16, 29 \& 42) as well as the net External balance (Cell 66).
Thus, Corresponding Figures are used for cells that are assumed to be identical.

%%%%%%%%%%%%%%%%%%%%%%%%%%%%%%%%%%%%%%%%%%%%%%%%%%%%%%%%%%
%SECTION
%%%%%%%%%%%%%%%%%%%%%%%%%%%%%%%%%%%%%%%%%%%%%%%%%%%%%%%%%%
\subsection{Data}
\label{sec:2.4.3}

\bigskip

The data used in the construction of the IncExp Accounts are derived from either UK or Scottish Government sources and are publicly available. Figure \ref{fig:2.4.1} shows the volume of data taken from the main data sources used. Each cell of the IncExp Account is deconstructed and the source of each component is identified \footnote{Note that IncExp Accounts entries used to calculate other the cells within the Accounts, for example (Cell 10), are not counted as a separate entry again. Also, the Shares (see Equations \ref{eq:2.5.80} to \ref{eq:2.5.87}) are not included as separate data entries, since these are solely used to transform UK data entries and are not used as separate individual items in the calculation of the Accounts.}. The majority of the data used for the IncExp Accounts is given in calendar year format, and each individual entry in that format, which is used in the calculation of a cell for the Accounts is counted as one entry. Data in financial year format is counted as one entry after it has been annualised (see Section \ref{sec:2.4.2} for further details). The total of the individual entries used for the derivation of the IncExp Accounts is then used to calculate the share of where the data for the Accounts originates from (see Figure \ref{fig:2.4.1}).

%IMAGE
\bigskip
\begin{figure}[H] \caption{Shares of data sources in Income and Expenditure Accounts}
\center{\includegraphics[width=1.0\linewidth]{incexpsources}} \label{fig:2.4.1}
\end{figure}
\bigskip
%END  

As an example, (Cell 5) is calculated by adding the annualised Total Social Protection payments, which is counted as one entry, and then UK Public Dividend payments. The latter is a summation of dividend payments from non-financial corporations, central government and local government, thus three entries are counted here, and the sum is multiplied by two shares, which are both not counted as separate entries. In this example, therefore four total entries are counted, which are taken from two separate sources (a share of 1/4 and 3/4 respectively).

\bigskip

Figure \ref{fig:2.4.1} shows that the largest source of data for the IncExp Accounts originates from the 2009 IO Tables for Scotland (with 31\%). The other two major data sources are depicted as GERS (with 27\%) and the ONS Blue Book (with 25\%). The total volume of UK data sources is calculated at 34\%, which is the sum of the shares of Blue Book, Public Expenditure Statistical Analysis (PESA) and Other UK Government \footnote{The shares are 25\%, 5\% and 4\% respectively.}

\bigskip

In order to transform UK data for the Scottish IncExp Accounts, three different Scottish shares are used. These shares are the GDP share (at 8.22\%), the population share (at 8.41\%) and the households share (9\%). These shares are all close in total value, however, theoretical considerations favour different shares for specific UK data as is outlined here. 

\bigskip

First, the GDP share is applied numerous times to transform UK data to Scottish data. For example, Governmental and Corporate transfers payments \cite{ONS2011c} are multiplied by the GDP share following the framework set out by \cite{Hermannsson2010a}. Second, the population share is used to transform PESA data on governmental expenditure \cite{HMTR2012}, which is in line with the methodology applied in GERS \cite{ScotGov2013b} for transforming PESA data sor Scotland. Third, the household share is applied to transform UK Dividend Payments to a Scottish figure, which is in line with UK calculations of transforming total dividend payments to the household level \cite{ONS2011c}.

\bigskip

The IncExp Accounts use a large share of Scottish-origin data, and any UK data figures are transformed using appropriate Scottish shares (as outlines above). Nevertheless, increasing the share of Scottish data would be desirable, for example minimising the share of Blue Book \cite{ONS2011c} data used. Another area for future improvement are Balancing Items, which make up 7\% of the total individual entries for the IncExp Accounts (see Figure \ref{fig:2.4.1}). Although these items are needed to balance any deviations between the income and the expenditure sides of the main transactors (as outlined in Section \ref{sec:2.4.2}), there is no external estimate for these figures at hand. The following Section gives a detailed breakdown of the individual cell calculations for the IncExp Accounts.


%%%%%%%%%%%%%%%%%%%%%%%%%%%%%%%%%%%%%%%%%%%%%%%%%%%%%%%%%%
%SECTION
%%%%%%%%%%%%%%%%%%%%%%%%%%%%%%%%%%%%%%%%%%%%%%%%%%%%%%%%%%
\newpage
\section{Income and Expenditure Accounts - Methodology}
\label{sec:2.5}

\bigskip
\begin{center}
\textbf{\LARGE Households}
\end{center}

\begin{enumerate}


\item \textbf {Income}

\bigskip

The Household income entry is derived from the latest revised figures of Scottish Gross Disposable Household Income (GDHI) for 2009 \cite{ONS2013a}. This data is obtained for Scotland at NUTS2 level covering the variables listed in Table \ref{tab:2.4.1}. The total Household Income figure of \textsterling107,877m is obtained by summing up Operating surplus$/$Mixed income (\textsterling9,437m), Compensation of employees (\textsterling64,645m), Property income received minus paid (\textsterling8,485m - \textsterling551m), Imputed social contributions$/$Social benefits received (\textsterling23,559m), and Other current transfers received minus Other current transfers paid (\textsterling5,102m - \textsterling2,800m).    

\bigskip

\begin{table}[H] \caption{Scottish Gross Disposable Household Income (GDHI) in \textsterling million by component}
\bigskip \begin{scriptsize} \begin{centering} \begin{doublespacing}
% INSERT
    \begin{tabular}{lr}
    \toprule
    Operating surplus$/$Mixed income &          9,437  \bigstrut[t]\\
    Compensation of employees &        64,645  \\
    Property income, received &          8,485  \\
    Primary resources total &        82,566  \bigstrut[b]\\
    Property income, paid &             551  \bigstrut[t]\\
    Primary uses total &             551  \\
    Balance of primary incomes &        82,015  \\
    Imputed social contributions/Social benefits received &        23,559  \bigstrut[b]\\
    Other current transfers, received &          5,102  \bigstrut[t]\\
    Secondary resources total &        28,663  \\
    Current taxes on income, wealth etc. &        13,893  \\
    Social contributions/Social benefits paid &        17,678  \bigstrut[b]\\
    Other current transfers, paid &          2,800  \bigstrut[t]\\
    Secondary uses total &        34,370  \\
    Balance of secondary income & -        5,708  \\
    Gross Disposable Income &        76,307  \bigstrut[b]\\
\bottomrule \end{tabular}%  
\bigskip \begin{flushright} Data sourced from: \cite{ONS2013a}\end{flushright} \label{tab:2.5.1} 
\end{doublespacing} \end{centering} \end{scriptsize} \end{table} \bigskip

\bigskip




\begin{equation}
\begin{split}
\text{Income} =
\text{Total Household Income}_\text{GDHI}
\end{split} \label{eq:2.5.1}
\end{equation}

\begin{equation} \nonumber
110677 = 110677
\end{equation}\\

\item \textbf {Income from Employment}

This is the ``Total intermediate demand'' \text{|| }``Compensation of employees'' from the IO Tables. [Source: Scottish Government (2013a)] \cite{ScottishGovernment2013a}

\begin{equation}
\begin{split}
\text{Income from Employment} =  \\ \\
(\text{Total Intermediate Demand}\|\text{Compensation of Employees})
\end{split} \label{eq:2.5.2}
\end{equation}

\begin{equation} \nonumber
63561 = 63561
\end{equation}\\


\item \textbf {Profit Income (OVA)}

\bigskip

This entry requires that the Gross Operating Surplus for Scotland is identified. Yet, as shown in Table \ref{tab:2.4.1}, data for Scotland is only available as an aggregate comprising of Operating surplus and Mixed income equal in total to \textsterling9,437m. Therefore, this figure has to be disaggregated to identify the Gross Operating Surplus component. This is estimated by using shares derived from 1999 GDHI data which reports these figures individually. There are no alternative datasets available that would allow for a better estimation of Scottish Gross Operating Surplus for 2009. 

\bigskip

Table \ref{tab:2.5.2} illustrates this process. First, the GDHI data for 1999 is obtained \shortcite{Hermannsson2010a}. Next, the the GDHI components are listed. Last, using the Gross Operating Surplus and Gross mixed income shares derived from 1999, the 2009 figures are disaggregated (i.e. \textsterling9437m * (\textsterling3413m / \textsterling3413m + \textsterling2677m) = \textsterling5289m and \textsterling9437 * (\textsterling2677m / \textsterling2677m + \textsterling3413m) = \textsterling4148m). This process yields the required Gross Operating Surplus estimate for Scotland of \textsterling5,289m. Thus, 2009 data (the control total) is disaggregate by using 1999 shares to yield the necessary variables.      

\bigskip

\begin{table}[H] \caption{Scottish Gross Disposable Household Income (GDHI) in \textsterling million by component}
\bigskip \begin{scriptsize} \begin{centering} \begin{doublespacing}
% INSERT
    \begin{tabular}{lrrr}
        \toprule
          & 1999  & 2009  & 2009 using shares \\
        \hline      
    Total household income &              71,296  &               107,877  &               107,877  \\
    Gross operating surplus &                 3,413  &                   9,437  &                   5,289  \\
    Gross mixed income &                 2,677  &   -    &                   4,148  \\
    Compensation of employees &              40,593  &                 64,645  &                 64,645  \\
    Net property income &                 6,591  &                   7,934  &                   7,934  \\
    All pensions &                 8,961  &                 23,559  &                 13,886  \\
   Other social benefits &                 6,242  &   -    &                   9,673  \\
    Net other income &                 2,820  &                   2,302  &                   2,302  \\
    \hline
    Total household disposable income &              48,931  &                 76,307  &                 76,307  \\
\bottomrule \end{tabular}%  
\bigskip \begin{flushright} Data sourced from: \cite{ONS2013a} and \shortcite{Hermannsson2010a}\end{flushright} \label{tab:2.4.2} 
\end{doublespacing} \end{centering} \end{scriptsize} \end{table} \bigskip



\begin{equation}
\begin{split}
\text{Profit Income} = \text{Gross Operating Surplus}_\text{GDHI}
\end{split} \label{eq:2.5.3}
\end{equation}

\begin{equation} \nonumber
5289 = 5289
\end{equation}\\


\item \textbf {Income from Corporations}

The income households receive from corporations is the sum of Capital Gains, then actual wages received and lastly any unallocated income calculated in the Income Accounts for Households. 
First, taking the Capital Gains Tax receipts as presented in GERS and dividing it by the fixed  Capital Gains Tax Rate for 2008-10 (at 18\%), gives an estimate of the actual monetary value of the capital gain received by Scottish households for 2009 \cite{ScotGov2013b,HMRC2013}.
Second, the total income (wages) received by households from corporations is added for the total figure. This comprises multiplying the share of Scottish GDP (see Equation \ref{eq:2.5.85}) , by the total of ``UK Private Dividends'' paid out by private non-financial corporations in the UK. In turn this figure is then multiplied by the average (figures are only available for 2008 and 2010) of an individual`s share of total equity on a UK basis, which is used to distinguish the dividend payments received by private shareholders versus, for example, funds \cite{ONS2011c}. Further, this part of the income figure is comprised of adding an estimate of the ``Total Private Pensions'' received by Scottish households to the above as well as household`s ``Net Other Income'' from the GDHI \cite{ONS2012}. 
Third, Households' Unallocated Income (see Cell 9) is added in order to balance this part of the Accounts.  


\begin{equation}
\begin{split}
\text{Income from Corporations} =  \\ \\
\text{Total Household Income from Corporations}\\
+\text{Household Income from Capital Gains}\\
+\text{Mixed and Prop Income Unallocated}_\text{IncExp}\\
\end{split} \label{eq:2.5.4}
\end{equation}

\begin{equation} \nonumber
17904 = 15558+1478+869
\end{equation}

\begin{center}
\line(1,0){250}
\end{center}

\textit{where}

\begin{equation} 
\begin{split}
\text{Total Household Income from Corporations} = \\ 
(\text{Scottish GDP Share}*\text{Total UK Private Dividend Payments})\\
*(\text{Individual Share of Total Equity} + \text{Total Private Pension}\\
+ \text{Net Other Income})
\end{split} \label{eq:2.5.5}
\end{equation} 

\begin{equation} \nonumber
15558 = (8.22\% * 85816)*((\frac{10.2\%+11.5\%}{2})+9691+5102)
\end{equation}

\begin{center}
\line(1,0){250}
\end{center}


\begin{equation} 
\begin{split}
\text{Household Income from Capital Gains} = \\ 
(1/4 *\text{Households' Captial Gains Tax Payments}_\text{08-09}\\
+ 3/4 *\text{Households' Captial Gains Tax Payments}_\text{09-10})\\
\div \text{Capital Gains Tax Rate}
\end{split} \label{eq:2.5.6}
\end{equation} 

\begin{equation} \nonumber
1478 = (1/4*572+3/4*164)\div 18\%
\end{equation}

\begin{center}
\line(1,0){250}
\end{center}


\begin{equation} 
\begin{split}
\text{Mixed and Prop Income Unallocated} = \\ 
(\text{Total Household Income}_\text{GDHI}-\text{Total Household Income}_\text{IncExp})\\
\end{split} \label{eq:2.5.7}
\end{equation} 

\begin{equation} \nonumber
869 = 110677-109808
\end{equation}\\

\item \textbf {Income from Government}

The first part of this figure is the annualised ``Social Protection Payments'' to Scottish households \cite{ScotGov2013b} and the second one is the ``Public Dividend Payments'' received by Scottish households \cite{ONS2011c}. The latter is calculated in accordance with the methodology outlined above for ``Private Dividend Payments''. The dividend payments are sourced from non-financial corporations, Central Government and Local Government accounts and are multiplied by the Scottish GDP share as well as the average individual`s share of total equity and further multiplied by the UK Public Dividend payments.  

\begin{equation}
\begin{split}
\text{Income from Government} =  \\ \\
(1/4*\text{Total Social Protection}_\text{08-09}\\
+3/4*\text{Total Social Protection}_\text{09-10})\\
+(\text{Scottish GDP Share} \\
*(\text{UK Public Dividends}_\text{Non-Financial Corporations}\\
+\text{UK Public Dividends}_\text{Central Government}\\
+\text{UK Public Dividends}_\text{Local Government})\\
*((\text{Individual's Share of Total Equity}_\text{2008}\\
+\text{Individual's Share of Total Equity}_\text{2009})\div 2))
\end{split} \label{eq:2.5.8}
\end{equation}


\begin{equation} \nonumber
\begin{split}
19835 = (1/4*18653+3/4*20193)\\
+(8.22\%*(25+2214+772)*((10.2\%+11.5\%)\div 2))
\end{split}
\end{equation}\\


\item \textbf {Transfers from RUK}

These transfers are calculated by first, taking the total figure of dividends paid to Scottish households. This figure is calculated by using the share of Scottish Households of total UK Households (see Equation \ref{eq:2.5.87}) and multiplying it by ``Total RUK Dividends'' paid to households \cite{ONS2011c}. The latter figure is based on the average individual`s share of total equity multiplied by the difference between Total UK- and Total Scottish- private dividends in order to obtain the RUK dividend payments to Households in Scotland \cite{ONS2011a,ONS2011b}. 
Second, this is then added to the difference of the ``Compensation of Employees'' according to the GDHI estimates and the actual figure of income from employment as calculated for the Income and Expenditure Account (see Cell 2).



\begin{equation}
\begin{split}
\text{Transfers from RUK} =  \\ \\
\text{Total RUK Dividends to Scottish Households}\\
+(\text{Compensation of Employees}_\text{GDHI}\\
-\text{Income from Employment}_\text{IncExp})
\end{split} \label{eq:2.5.9}
\end{equation}

\begin{equation} \nonumber
1852 = 767+(64645-63561)
\end{equation}\\

\begin{center}
\line(1,0){250}
\end{center}

\textit{where}

\begin{equation}
\begin{split}
\text{Total RUK Dividends to Scottish Households}=\\
\text{Scottish Household Share}*\text{Total RUK Dividends to Households}
\end{split} \label{eq:2.5.10}
\end{equation}

\begin{equation}\nonumber
767=8.98\%*8546
\end{equation}


\item \textbf {Transfers from ROW}

The first part of this figure is calculated by multiplying UK employment income from ROW \cite{ONS2011c} with Scottish Share of Total Corporate OVA (see Equation \ref{eq:2.5.84}). Added to this is the Scottish share of UK GDP (see Equation \ref{eq:2.5.85}) multiplied with the Scottish household share of OVA for UK property and entrepreneurial income and multiplied by the actual amount of the ``UK Property and Entrepreneurial Income''. \cite{ScotGov2013a,ScotGov2013b}

\begin{equation}
\begin{split}
\text{Transfers from ROW} =  \\ \\
(\text{Scottish Share of UK Total OVA}*\\
\text{UK Employment Income from ROW})\\
+(\text{Scottish Household OVA}*\text{Scottish GDP Share of UK}\\
*\text{UK Property and Entrepreneurial Income})
\end{split} \label{eq:2.5.11}
\end{equation}


\begin{equation} \nonumber
2237 = (143588.31\%)+(169313*15\%*8.22\%)
\end{equation}\\

\item \textbf {Total Household Income}

Totals Figure: Summation of all of the above, excluding the total household income figure obtained from the GDHI (sum of cells 2 to 7).

\begin{equation}
\begin{split}
\text{Total Household Income} =  \\ \\
(\text{Income from Employment}^\text{Households}_\text{IncExp}\\
+\text{Profit Income (OVA)}^\text{Households}_\text{IncExp}\\
+\text{Income from Corporations}^\text{Households}_\text{IncExp}\\
+\text{Income from Government}^\text{Households}_\text{IncExp}\\
+\text{Transfers from RUK}^\text{Households}_\text{IncExp}\\
+\text{Transfers from ROW}^\text{Households}_\text{IncExp})
\end{split} \label{eq:2.5.12}
\end{equation}

\begin{equation} \nonumber
110677 = 63561+5289+17904+19835+1852+2237
\end{equation}\\


\item \textbf {Mixed and Prop Income Unallocated}

Balancing item equal to the difference of Household Income as presented in the GDHI \cite{ONS2012} and the sum of all income figures presented above. This figure gets added into the Income from Corporations (Cell 4) which results in this cell equalling zero, due to the two household income figures (see Cell 1 and 8) balancing now \cite{ONS2011b}.

\begin{equation}
\begin{split}
\text{Income Unallocated} =  \\ \\
\text{Income}^\text{Households}_\text{IncExp}\\
-\text{Income from Employment}^\text{Households}_\text{IncExp}
\end{split} \label{eq:2.5.13}
\end{equation}


\begin{equation} \nonumber
869 = 110677-109808
\end{equation}\\


\pagebreak

\item \textbf {Expenditure}

Totals Figure: Summation of figures below, from IO Expenditure to Payments to Capital (sum: 11 to 16).

\begin{equation}
\begin{split}
\text{Expenditure} =  \\ \\
(\text{IO Expenditure}^\text{Households}_\text{IncExp}\\
+\text{Payments to Corporations}^\text{Households}_\text{IncExp}\\
+\text{Payments to Government}^\text{Households}_\text{IncExp}\\
+\text{Payments to Capital}^\text{Households}_\text{IncExp}\\
+\text{Transfers from RUK}^\text{Households}_\text{IncExp}\\
+\text{Transfers from ROW}^\text{Households}_\text{IncExp})
\end{split} \label{eq:2.5.14}
\end{equation}

\begin{equation} \nonumber
110677 = 74138+9600+21379+5202+238+119
\end{equation}\\


\item \textbf {IO Expenditure}

This cell is made up of ``Households' Final Consumption Expenditure'' || ``Total intermediate consumption at basic prices'' plus ``Households Final Consumption Expenditure'' || ``Taxes less subsidies on products'' and ``Non-Profit Institutions Serving Households' Final Consumption Expenditure'' || ``Total intermediate consumption at basic prices'' plus ``Non-Profit Institutions Serving Households' Final Consumption Expenditure'' || ``Taxes less subsidies on products'' from the IO Tables \cite{ScotGov2013a}.

\begin{equation}
\begin{split}
\text{IO Expenditure} =  \\ \\
(\text{Final Cons. Expend.}_\text{Households}\|\text{Total Interm. Cons.})\\
+(\text{Final Cons. Expend.}_\text{NPISH}\|\text{Total Interm. Cons.})\\
+(\text{Final Cons. Expend.}_\text{Households}\|\text{Taxes less subsidies on products})\\
+(\text{Final Cons. Expend.}_\text{NPISH}\|\text{Taxes less subsidies on products})
\end{split} \label{eq:2.5.15}
\end{equation}

\begin{equation} \nonumber
74138 = 64890+6568+26803+0
\end{equation}\\


\item \textbf {Payments to Corporations}

Balancing Item: Taking the Total Expenditure (Cell 17) and subtracting the IO Expenditure, Payments to Government, Payments to Capital, Transfers to RUK and Transfers to ROW from it (Cells 11,14,15,16).

\begin{equation}
\begin{split}
\text{Payments to Corporations} =  \\ \\
\text{Total Expenditure}^\text{Households}_\text{IncExp}-\text{Transfers to ROW}^\text{Households}_\text{IncExp}\\
-\text{Transfers to RUK}^\text{Households}_\text{IncExp}-\text{Payments to Capital}^\text{Households}_\text{IncExp}\\
-\text{Payments to Government}^\text{Households}_\text{IncExp}-\text{IO Expenditure}^\text{Households}_\text{IncExp}
\end{split} \label{eq:2.5.16}
\end{equation}

\begin{equation} \nonumber
9600 = 110677-119-238-5202-21379-74138
\end{equation}\\


\item \textbf {Payments to Government}

These are the annualised tax payments by Scottish households to the central government. These taxes are: Income Tax, Capital Gains Tax, Inheritance Tax, Stamp Duties, Half Insurance Premium Tax, Council Tax and Social Security Contributions (NI) \cite{ScotGov2013b}.

\begin{equation}
\begin{split}
\text{Payments to Government} =  \\ \\
(1/4*(\text{Income Tax}_\text{08-09}+\text{Capital Gains Tax}_\text{08-09}\\
+(\text{Inheritance Tax}_\text{08-09}+\text{Stamp Duties}_\text{08-09}\\
+(\text{Half Insurance Premium Tax}_\text{08-09}+\text{Council Tax}_\text{08-09}\\
+(\text{Social Security Contributions}_\text{08-09}))\\
+(3/4*(\text{Income Tax}_\text{09-10}+\text{Capital Gains Tax}_\text{09-10}\\
+(\text{Inheritance Tax}_\text{09-10}+\text{Stamp Duties}_\text{09-10}\\
+(\text{Half Insurance Premium Tax}_\text{09-10}+\text{Council Tax}_\text{09-10}\\
+(\text{Social Security Contributions}_\text{09-10}))\\
\end{split} \label{eq:2.5.17}
\end{equation}

\begin{equation} \nonumber
\begin{split}
21379=(1/4*(10642+572+178+594+96+1960+7992))\\
+(3/4*(10364+164+146+516+95+1961+7915))
\end{split}
\end{equation}\\


\item \textbf {Transfers to RUK}

This figure is calculated using the methodology outlined for the cell below (Transfers to ROW - 16). It is assumed that the transfers paid to the RUK are twice as high as those paid to the ROW, and thus this cell is equal to Cell 16 times two.

\begin{equation}
\begin{split}
\text{Transfers to RUK} =  \\ \\
\text{Transfers to ROW}^\text{Households}_\text{IncExp}*2
\end{split} \label{eq:2.5.18}
\end{equation}

\begin{equation} \nonumber
238 = 119*2
\end{equation}\\


\item \textbf {Transfers to ROW}

This figure is made up of the amount of employee compensation that is paid to the ROW, i.e. the part that is deducted from GDP in order to arrive at GNP figures, times the share of Scottish OVA of Corporate Income (\ref{eq:2.5.84}) \cite{ONS2011c}.

\begin{equation}
\begin{split}
\text{Transfers to ROW} =  \\ \\
\text{UK Payments to ROW}*\text{Scottish Corporate Income OVA}
\end{split} \label{eq:2.5.19}
\end{equation}

\begin{equation} \nonumber
119 = 1435*8.31\%
\end{equation}\\


\item \textbf {Payments to Capital (Savings)}

This cell is calculated by assuming that households save a share of their total expenditure. Using the Household Saving Rate taken from the Scottish National Accounts Project (SNAP) for  2009\cite{ScotGov2013c}, this rate is then multiplied with the Total Expenditure (see Cell 18) .

\begin{equation}
\begin{split}
\text{Payments to Capital} =  \\ \\
\text{Total Household Income}^\text{Households}_\text{IncExp}\\
+\text{Household Savings Rate}_\text{SNAP}
\end{split} \label{eq:2.5.20}
\end{equation}

\begin{equation} \nonumber
5202 = 110677*0.047
\end{equation}\\


\item \textbf {Total Expenditure}

Corresponding Figure: Equal to the Total Household Income (9), since the assumption is made that total incomes for household are equal to their total expenditure.

\begin{equation}
\begin{split}
\text{Total Expenditure} =  \\ \\
\text{Total Household Income}^\text{Households}_\text{IncExp}
\end{split} \label{eq:2.5.21}
\end{equation}

\begin{equation} \nonumber
110677 = 110677
\end{equation}\\



\pagebreak

\begin{center}
\textbf{\LARGE Corporations}
\end{center}

\item \textbf {Income}

Totals Figure: Equal to all of the items below in this section (see Cells 19 to 23).

\begin{equation}
\begin{split}
\text{Income} =  \\ \\
\text{Profit Income}^\text{Corporations}_\text{IncExp}\\
+\text{Income from Households}^\text{Corporations}_\text{IncExp}\\
+\text{Income from Government}^\text{Corporations}_\text{IncExp}\\
+\text{Income from RUK}^\text{Corporations}_\text{IncExp}\\
+\text{Income from ROW}^\text{Corporations}_\text{IncExp}
\end{split} \label{eq:2.5.22}
\end{equation}

\begin{equation} \nonumber
56175 = 29456+9600+5191+5964+5964
\end{equation}\\

\item \textbf {Profit Income (OVA)}

Taking the ``Total Intermediate Demand'' – ``Gross Operating Surplus'', the OVA of both Households and Government (3 and 31) are deducted from it from it.  \cite{ScotGov2013a,ONS2011b}

\begin{equation}
\begin{split}
\text{Profit Income} =  \\ \\
\text{Total Intermediate Demand}\|\text{Gross Operating Surplus}\\
-\text{Profit Income}^\text{Households}_\text{IncExp}-\text{Profit Income}^\text{Government}_\text{IncExp}
\end{split} \label{eq:2.5.23}
\end{equation}

\begin{equation} \nonumber
29456 = 38441-5289-3697
\end{equation}\\


\item \textbf {Income from Households}

Corresponding Figure: Equal to Payments to Corporations under Household Expenditure (12).

\begin{equation}
\begin{split}
\text{Income from Households} =  \\ \\
\text{Payments to Corporations}^\text{Households}_\text{IncExp}
\end{split} \label{eq:2.5.24}
\end{equation}

\begin{equation} \nonumber
9600 = 9600
\end{equation}\\


\item \textbf {Income from Government}

Corresponding Figure: Equal to Payments to Corporations under Government Expenditure (39).

\begin{equation}
\begin{split}
\text{Income from Government} =  \\ \\
\text{Payments to Corporations}^\text{Government}_\text{IncExp}
\end{split} \label{eq:2.5.25}
\end{equation}

\begin{equation} \nonumber
5191 = 5191
\end{equation}\\


\item \textbf {Income from RUK}

Using the Scottish share of UK property and entrepreneurial income (see \ref{eq:2.5.82}), it is multiplied by the corporate share of OVA. One half of this figure is used for this cell and the other for the one below (23). \cite{ONS2011c}

\begin{equation}
\begin{split}
\text{Income from RUK} =  \\ \\
1/2*\text{Corporate OVA Share}\\
*\text{Scottish Share of UK Property and Entrepreneurial Income}
\end{split} \label{eq:2.5.26}
\end{equation}

\begin{equation} \nonumber
5964 = 84.8\%*14070*1/2
\end{equation}\\


\item \textbf {Income from ROW}

Other half of figure calculated in first part of 22.

\begin{equation}
\begin{split}
\text{Income from RUK} =  \\ \\
1/2*\text{Corporate OVA Share}\\
*\text{Scottish Share of UK Property and Entrepreneurial Income}
\end{split} \label{eq:2.5.27}
\end{equation}

\begin{equation} \nonumber
5964 = 84.8\%*14070*1/2
\end{equation}\\


\pagebreak

\item \textbf {Expenditure}

Totals Figure: cells below (25 to 29).

\begin{equation}
\begin{split}
\text{Expenditure} =  \\ \\
\text{Payments to Households}^\text{Corporations}_\text{IncExp}\\
+\text{Payments to Government}^\text{Corporations}_\text{IncExp}\\
+\text{Transfers to RUK}^\text{Corporations}_\text{IncExp}\\
+\text{Transfers to ROW}^\text{Corporations}_\text{IncExp}\\
+\text{Payments to Capital}^\text{Corporations}_\text{IncExp}
\end{split} \label{eq:2.5.28}
\end{equation}

\begin{equation} \nonumber
56175 = 17904+5248+3768+4560+24695
\end{equation}\\


\item \textbf {Payments to Households}

Corresponding Figure: Equal to Household Income from Corporations (4).

\begin{equation}
\begin{split}
\text{Payments to Households} =  \\ \\
\text{Income from Corporations}^\text{Households}_\text{IncExp}
\end{split} \label{eq:2.5.29}
\end{equation}

\begin{equation} \nonumber
17904 = 17904
\end{equation}\\


\item \textbf {Payments to Government}

These are the annualised corporate taxes: Corporation Tax, (Windfall Tax) Half Insurance Premium Tax, Landfill Tax, Non-Domestic Rates, Other Taxes and Royalties, Interest and Dividends \cite{ScotGov2013b}

\begin{equation}
\begin{split}
\text{Payments to Government} =  \\ \\
(1/4*(\text{Corporation Tax}_\text{08-09}+\text{Half Insurance Premium Tax}_\text{08-09}\\
+(\text{Landfill Tax}_\text{08-09}+\text{Non-Domestic Rates}_\text{08-09}\\
+(\text{Other Taxes and Royalties}_\text{08-09}+\text{Interest and Dividends}_\text{08-09}\\
+(3/4*(\text{Corporation Tax}_\text{09-10}+\text{Half Insurance Premium Tax}_\text{09-10}\\
+(\text{Landfill Tax}_\text{09-10}+\text{Non-Domestic Rates}_\text{09-10}\\
+(\text{Other Taxes and Royalties}_\text{09-10}+\text{Interest and Dividends}_\text{09-10}\\
\end{split} \label{eq:2.5.30}
\end{equation}

\begin{equation} \nonumber
\begin{split}
5248 = (1/4*(2841+96+82+1736+250+608))\\
+(3/4(2680+95+85+1822+212+233))
\end{split}
\end{equation}\\


\item \textbf {Transfers to RUK}

Equal to OVA repatriated to RUK (see \ref{eq:2.5.80}). \cite{ScotGov2012}\\

\begin{equation}
\begin{split}
\text{Transfers to RUK} =  \\ \\
\text{Share of OVA Repatriated to RUK}*\text{Profit Income}^\text{Corporations}_\text{IncExp}
\end{split} \label{eq:2.5.31}
\end{equation}

\begin{equation} \nonumber
3768 = 13\%829456
\end{equation}\\


\item \textbf {Transfers to ROW}

Equal to OVA repatriated to ROW (see \ref{eq:2.5.81}). \cite{ScotGov2012}

\begin{equation}
\begin{split}
\text{Transfers to ROW} =  \\ \\
\text{Share of OVA Repatriated to ROW}*\text{Profit Income}^\text{Corporations}_\text{IncExp}
\end{split} \label{eq:2.5.32}
\end{equation}

\begin{equation} \nonumber
4560 = 15\%*829456
\end{equation}\\


\item \textbf {Payments to Capital (Savings)}

Balancing Item: This figure is derived by summing up the ``Gross Fixed Capital Formation'' (GFCF) for all Public Sectors in the IO Tables and then deducting the sum of the ``Taxes less subsidies on production'' for these sectors. The Public Sectors are: Water and Sewerage, Public Administration and Defence, Education, Health, Residential Care and Social Work. \cite{ScotGov2013a}

\begin{equation}
\begin{split}
\text{Payments to Capital} =  \\ \\
\text{Income}^\text{Corporations}_\text{IncExp}\\
-\text{Payments to Households}^\text{Corporations}_\text{IncExp}\\
-\text{Payments to Government}^\text{Corporations}_\text{IncExp}\\
-\text{Transfers to RUK}^\text{Corporations}_\text{IncExp}\\
-\text{Transfers to ROW}^\text{Corporations}_\text{IncExp}
\end{split} \label{eq:2.5.33}
\end{equation}

\begin{equation} \nonumber
24695 = 56175-17904-5248-3768-4560
\end{equation}\\


\pagebreak

\begin{center}
\textbf{\LARGE Government}
\end{center}

\item \textbf {Income}

Totals Figure: Sum of cells below (31 to 35).

\begin{equation}
\begin{split}
\text{Income} =  \\ \\
\text{Profit Income}^\text{Government}_\text{IncExp}\\
+\text{Net Commodity Tax}^\text{Government}_\text{IncExp}\\
+\text{Income from Households}^\text{Government}_\text{IncExp}\\
+\text{Income from Corporations}^\text{Government}_\text{IncExp}
\end{split} \label{eq:2.5.34}
\end{equation}

\begin{equation} \nonumber
63530 = 3697+13165+21379+5248+10041
\end{equation}\\


\item \textbf {Profit Income (OVA)}

Equal to ``Taxes less subsidies on production'' for all public sectors (see 30).  \cite{ScotGov2013a}

\begin{equation}
\begin{split}
\text{Profit Income} =  \\ \\
\text{Water and Sewerage}\|\text{Gross Operating Surplus}\\
+\text{Public Administration and Defence}\|\text{Gross Operating Surplus}\\
+\text{Education}\|\text{Gross Operating Surplus}\\
+\text{Health}\|\text{Gross Operating Surplus}\\
+\text{Residential Care}\|\text{Gross Operating Surplus}\\
+\text{Social Work}\|\text{Gross Operating Surplus}
\end{split} \label{eq:2.5.35}
\end{equation}

\begin{equation} \nonumber
3697 = 710+865+463+817+590+253
\end{equation}\\


\item \textbf {Net Commodity Taxes}

This cell is the sum of ``Total Intermediate Deman''” – ``Taxes less subsidies on production'' and ``Total Demand for Products'' – ``Taxes less subsidies on products''. \cite{ScotGov2013a}\\

\begin{equation}
\begin{split}
\text{Net Commodity Taxes} =  \\ \\
\text{Total Intermediate Demand}\|\text{Taxes less Subsidies on Production}\\
+\text{Total Demand for Products}\|\text{Taxes less Subsidies on Products}\\
\end{split} \label{eq:2.5.36}
\end{equation}

\begin{equation} \nonumber
13165 = 1232+11933
\end{equation}\\


\item \textbf {Income from Households}

Corresponding Figure: Equal to Payments to Government under Household Expenditure (13).

\begin{equation}
\begin{split}
\text{Income from Households} =  \\ \\
\text{Payments to Government}^\text{Households}_\text{IncExp}
\end{split} \label{eq:2.5.37}
\end{equation}

\begin{equation} \nonumber
21379 = 21379
\end{equation}\\


\item \textbf {Income from Corporations}

Corresponding Figure: Equal to Payments to Government under Corporations Expenditure (26).

\begin{equation}
\begin{split}
\text{Income from Coporations} =  \\ \\
\text{Payments to Government}^\text{Corporations}_\text{IncExp}
\end{split} \label{eq:2.5.38}
\end{equation}

\begin{equation} \nonumber
5248 = 5248
\end{equation}\\


\item \textbf {Income from RUK}

Balancing Item: Total Gov. Income Balancing Total (36) minus the sum of Profit Income, Net Commodity Taxes, Income from Households and Income from Corporations (31 to 34).\\

\begin{equation}
\begin{split}
\text{Income from RUK} =  \\ \\
\text{Total Government Income Balancing}\\
-\text{Profit Income}^\text{Government}_\text{IncExp}\\
-\text{Net Commodity Taxes}^\text{Government}_\text{IncExp}\\
-\text{Income from Households}^\text{Government}_\text{IncExp}\\
-\text{Income from Corporations}^\text{Government}_\text{IncExp}
\end{split} \label{eq:2.5.39}
\end{equation}

\begin{equation} \nonumber
20041 = 63530-3697-13165-21379-5248
\end{equation}\\


\item \textbf {Total Government Income Balancing Total}

Corresponding Figure: Equal to Total Government Expenditure Balancing Total (43).\\

\begin{equation}
\begin{split}
\text{Total Government Income} =  \\ \\
\text{Total Government Expenditure Balancing Total}^\text{Government}_\text{IncExp}
\end{split} \label{eq:2.5.40}
\end{equation}

\begin{equation} \nonumber
63530 = 63530
\end{equation}\\



\pagebreak

\item \textbf {Expenditure}

Totals Figure: Summation for cells below (38 to 42).\\

\begin{equation}
\begin{split}
\text{Expenditure} =  \\ \\
\text{IO Expenditure}^\text{Government}_\text{IncExp}\\
+\text{Payments to Corporations}^\text{Government}_\text{IncExp}\\
+\text{Payments to Households}^\text{Government}_\text{IncExp}\\
+\text{Transfers to RUK}^\text{Government}_\text{IncExp}\\
+\text{Payments to Capital}^\text{Government}_\text{IncExp}
\end{split} \label{eq:2.5.41}
\end{equation}

\begin{equation} \nonumber
63530 = 30017+5191+19835+8368+119
\end{equation}\\


\item \textbf {IO Expenditure}

This is the ``Central Government'' and ``Local Governments'' – ``Total intermediate consumption at basic prices''. \cite{ScotGov2013a}

\begin{equation}
\begin{split}
\text{IO Expenditure} =  \\ \\
\text{Central Government}\|\text{Total Intermediate Consumption at basic Prices}\\
+\text{Local Government}\|\text{Total Intermediate Consumption at basic Prices}
\end{split} \label{eq:2.5.42}
\end{equation}

\begin{equation} \nonumber
30017 = 19462+10555
\end{equation}\\


\item \textbf {Payments to  Corporations}

Balancing Item:  Total Government Expenditure Balancing Total (44) minus IO Expenditure, Payments to Households, Transfers to RUK and Payments to Capital (Savings) (38, 40, 41, 42).

\begin{equation}
\begin{split}
\text{Payments to Corporations} =  \\ \\
\text{Total Government Expenditure Balancing Total}^\text{Government}_\text{IncExp}\\
-\text{IO Expenditure}^\text{Government}_\text{IncExp}\\
+\text{Payments to Households}^\text{Government}_\text{IncExp}\\
+\text{Transfers to RUK}^\text{Government}_\text{IncExp}\\
+\text{Payments to Capital}^\text{Government}_\text{IncExp}
\end{split} \label{eq:2.5.43}
\end{equation}

\begin{equation} \nonumber
5191 = 63530-30017-19835-8368-119
\end{equation}\\


\item \textbf {Payments to Households}

Corresponding Figure: Income from Government from the Household Income Accounts (5).\\

\begin{equation}
\begin{split}
\text{Payments to Households} =  \\ \\
\text{Income from Government}^\text{Households}_\text{IncExp}
\end{split} \label{eq:2.5.44}
\end{equation}

\begin{equation} \nonumber
19835 = 19835
\end{equation}\\


\item \textbf {Transfers to RUK}

This is the annualised estimated non-identifiable Government Expenditure, which is based on the Scottish population share of the UK Total non-identifiable public spending. \cite{ScotGov2013b}

\begin{equation}
\begin{split}
\text{Transfers to RUK} =  \\ \\
1/4*\text{Estimated Non-Identifiable Expenditure}_\text{08-09}\\
+3/4\text{Estimated Non-Identifiable Expenditure}_\text{09-10}
\end{split} \label{eq:2.5.45}
\end{equation}

\begin{equation} \nonumber
8368 = 1/4*8174+3/4*8432
\end{equation}\\


\item \textbf {Payments to Capital (Savings)}

This is the sum of ``Gross Fixes Capital Formation'' for all Public Sectors, which is then subtracted by ``Taxes less subsidies on production'' for these sectors. \cite{ScotGov2013a}\\

\begin{equation}
\begin{split}
\text{Payments to Capital} =  \\ \\
(\text{Gross Fixed Capital Formation}\|\text{Water and Sewerage}\\
+\text{Gross Fixed Capital Formation}\|\text{Public Administration and Defence}\\
+\text{Gross Fixed Capital Formation}\|\text{Education}\\
+\text{Gross Fixed Capital Formation}\|\text{Health}\\
+\text{Gross Fixed Capital Formation}\|\text{Residential Care}\\
+\text{Gross Fixed Capital Formation}\|\text{Social Work})\\
-(\text{Water and Sewerage}\|\text{Taxes less Subsidies on Production}\\
+\text{Public Administration and Defence}\|\text{Taxes less Subsidies on Production}\\
+\text{Education}\|\text{Taxes less Subsidies on Production}\\
+\text{Health}\|\text{Taxes less Subsidies on Production}\\
+\text{Residential Care}\|\text{Taxes less Subsidies on Production}\\
+\text{Social Work}\|\text{Taxes less Subsidies on Production})
\end{split} \label{eq:2.5.46}
\end{equation}

\begin{equation} \nonumber
\begin{split}
119 = (1+174+7+0+0+1)\\
-(28+0+18+11+3+4)
\end{split}
\end{equation}\\


\item \textbf {Total Government Expenditure Balancing Total}

This is the annualised ``Total Identifiable Expenditure'' of the Scottish Government plus the non-identifiable estimate (see 41). Then, the annualised ``Total managed expenditure'', ``Total Identifiable''- and ``Total non-identifiable Expenditure'' of the UK is multiplied by the Scottish population share of the UK Total population and then taken off the two former sums of Public Sector spending in Scotland. \cite{HMTR2012,ONS2011a}\\

\begin{equation}
\begin{split}
\text{Total Government Expenditure} =  \\ \\
(1/4 * \text{Total Identifiable Expenditure}_{08-09} \\
+ 3/4 * \text{Total Identifiable Expenditure}_{09-10})\\
+ (1/4 * \text{Total Non-Identifiable Expenditure}_{08-09} \\
+ 3/4 * \text{Total Non-Identifiable Expenditure}_{09-10}) \\
( 1/4 * \text{Scot. Pop. Share} * ( \text{Total Man. Exp.}^{UK}_{08-09} \\
- \text{Total Ident. Exp.}^{UK}_{08-09} \\
- \text{Total Man. Non-Ident.}^{UK}_{08-09})) \\
( 1/4 * \text{Scot. Pop. Share} * ( \text{Total Man. Exp.}^{UK}_{09-10} \\
- \text{Total Ident. Exp.}^{UK}_{09-10} \\
- \text{Total Man. Non-Ident.}^{UK}_{09-10}))
\end{split} \label{eq:2.5.47}
\end{equation}


\begin{equation} \nonumber
\begin{split}
63530 = (1/4*(50779+8174))+(3/4*(53617+8432))\\
+(1/4*8.41\%*(629745-515734-87697))\\
+(3/4*8.41\%*(670150-559134-84021))
\end{split}
\end{equation}\\


\pagebreak


\begin{center}
\textbf{\LARGE Capital}
\end{center}

\item \textbf {Income}

Totals Figure: Sum of cells below (45 to 48).

\begin{equation}
\begin{split}
\text{Income} =  \\ \\
\text{Households}^\text{Capital}_\text{IncExp}\\
+\text{Corporations}^\text{Capital}_\text{IncExp}\\
+\text{Government}^\text{Capital}_\text{IncExp}\\
+\text{RUK/ROW}^\text{Capital}_\text{IncExp}
\end{split} \label{eq:2.5.48}
\end{equation}

\begin{equation} \nonumber
19930 = 5202+24695+119+(-10086)
\end{equation}\\


\item \textbf {Households)}

Corresponding Figure: Payments to Capital of the Household Expenditure Account (14).

\begin{equation}
\begin{split}
\text{Households} =  \\ \\
\text{Payments to Capital}^\text{Households}_\text{IncExp}
\end{split} \label{eq:2.5.49}
\end{equation}

\begin{equation} \nonumber
5202 = 5202
\end{equation}\\


\item \textbf {Corporate}

Corresponding Figure: Payments to Capital (savings) of the Corporation Expenditure Account (29).

\begin{equation}
\begin{split}
\text{Corporate} =  \\ \\
\text{Payments to Capital}^\text{Corporations}_\text{IncExp}
\end{split} \label{eq:2.5.50}
\end{equation}

\begin{equation} \nonumber
24695 = 24695
\end{equation}\\


\item \textbf {Government}

Corresponding Figure: Payments to Capital (savings) of the Government Expenditure Account (42).

\begin{equation}
\begin{split}
\text{Government} =  \\ \\
\text{Payments to Capital}^\text{Government}_\text{IncExp}
\end{split} \label{eq:2.5.51}
\end{equation}

\begin{equation} \nonumber
119 = 119
\end{equation}\\


\item \textbf {RUK/ROW}

Corresponding Figure: Surplus/Deficit of the External Expenditure Account (66).

\begin{equation}
\begin{split}
\text{RUK/ROW} =  \\ \\
\text{Total Income}^\text{External}_\text{IncExp}\\
-\text{Total Expenditure}^\text{External}_\text{IncExp}
\end{split} \label{eq:2.5.52}
\end{equation}

\begin{equation} \nonumber
-10086 = 90808-100894
\end{equation}\\


\pagebreak

\item \textbf {Expenditure}

Corresponding Figure: IO Expenditure (50).
    
\begin{equation}
\begin{split}
\text{Expenditure} =  \\ \\
\text{IO Expenditure}^\text{Capital}_\text{IncExp}
\end{split} \label{eq:2.5.53}
\end{equation}

\begin{equation} \nonumber
19930 = 19930
\end{equation}\\


\item \textbf {IO Expenditure}

This is the sum of ``Total Gross Capital Formation'' – ``Total intermediate consumption at basic prices'' and ``Total Gross Capital Formation'' – ``Taxes less subsidies on products''. \cite{ScotGov2013a}

\begin{equation}
\begin{split}
\text{IO Expenditure} =  \\ \\
\text{Total Gross Capital Formation}\|\text{Total Interm. Consumption at Basic Prices}\\
+\text{Total Gross Capital Formation}\|\text{Taxes Less Subsidies on Products}
\end{split} \label{eq:2.5.54}
\end{equation}

\begin{equation} \nonumber
19930 = 18453+1495
\end{equation}\\



\pagebreak


\begin{center}
\textbf{\LARGE External}
\end{center}

\item \textbf {UK Income from Scotland}

Totals Figure: This is the sum of the two cells below: Goods \& Services and Transfers (52 \& 53).\\

\begin{equation}
\begin{split}
\text{UK Income from Scotland} =  \\ \\
\text{Goods \& Services}^\text{External}_\text{IncExp}\\
+\text{Transfers}^\text{External}_\text{IncExp}
\end{split} \label{eq:2.5.55}
\end{equation}

\begin{equation} \nonumber
67133 = 54759+12374
\end{equation}\\


\item \textbf {Goods \& Services}

This is the “Total Demand for Products” from RUK. \cite{ScotGov2013a}

\begin{equation}
\begin{split}
\text{Goods \& Services} =  \\ \\
\text{Total Demand for Products}\|\text{Imports from Rest of UK}
\end{split} \label{eq:2.5.56}
\end{equation}

\begin{equation} \nonumber
54759 = 54759
\end{equation}\\


\item \textbf {Transfers}

This is the sum of: ``Transfers to RUK'' from the Household Expenditure Account, ``Transfers to RUK'' from the Corporations Expenditure Account and the ``Transfers to RUK'' from the Government Expenditure Account (6, 22, 35).\\

\begin{equation}
\begin{split}
\text{Transfers} =  \\ \\
\text{UK Income from Scotland} =  \\ \\
\text{Transfers to RUK}^\text{Households}_\text{IncExp}\\
+\text{Transfers to RUK}^\text{Corporations}_\text{IncExp}\\
+\text{Transfers to RUK}^\text{Government}_\text{IncExp}
\end{split} \label{eq:2.5.57}
\end{equation}

\begin{equation} \nonumber
12374 = 238+3768+8368
\end{equation}\\


\item \textbf {ROW Income from Scotland}

Totals Figure: This is the sum of the two cells below: Goods \& Services and Transfers (55 \& 56).\\

\begin{equation}
\begin{split}
\text{ROW Income from Scotland} =  \\ \\
\text{Goods \& Services}^\text{External}_\text{IncExp}\\
+\text{Transfers}^\text{External}_\text{IncExp}
\end{split} \label{eq:2.5.58}
\end{equation}

\begin{equation} \nonumber
23676 = 18997+4697
\end{equation}\\


\item \textbf {Goods \& Services}

This is the ``Total Demand for Products'' – ``ROW''. \cite{ScotGov2013a}

\begin{equation}
\begin{split}
\text{Goods \& Services} =  \\ \\
\text{Total Demand for Products}\|\text{Imports from ROW}
\end{split} \label{eq:2.5.59}
\end{equation}

\begin{equation} \nonumber
18997 = 18997
\end{equation}\\


\item \textbf {Transfers}

This is the sum of: ``Transfers to ROW'' from the Household Expenditure Account and ``Transfers to ROW'' from the Corporations Expenditure Account (7 \& 23).\\

\begin{equation}
\begin{split}
\text{Transfers} =  \\ \\
\text{Transfers to ROW}^\text{Households}_\text{IncExp}\\
+\text{Transfers to RUK}^\text{Corporations}_\text{IncExp}
\end{split} \label{eq:2.5.60}
\end{equation}

\begin{equation} \nonumber
4679 = 119+4560
\end{equation}\\


\item \textbf {Total Income}

Totals Figure: This is the sum of the two cells above: ``UK income from Scotland'' and ``ROW income from Scotland'' (51 \& 54).\\

\begin{equation}
\begin{split}
\text{Total Income} =  \\ \\
\text{UK Income from Scotland}^\text{External}_\text{IncExp}\\
+\text{ROW Income from Scotland}^\text{External}_\text{IncExp}
\end{split} \label{eq:2.5.61}
\end{equation}

\begin{equation} \nonumber
90808 = 67133+23676
\end{equation}\\



\pagebreak

\item \textbf {UK Expenditure in Scotland}

Totals Figure: This is the sum of the two cells below: Goods \& Services and Transfers (59 \& 60).\\

\begin{equation}
\begin{split}
\text{UK Expenditure in Scotland} =  \\ \\
\text{Goods \& Services}^\text{External}_\text{IncExp}\\
+\text{Transfers}^\text{External}_\text{IncExp}
\end{split} \label{eq:2.5.62}
\end{equation}

\begin{equation} \nonumber
70595 = 42739+27857
\end{equation}\\


\item \textbf {Goods \& Services}

This is the ``Total intermediate consumption at basic prices'' – ``Rest of UK exports''. \cite{ScotGov2013a}\\

\begin{equation}
\begin{split}
\text{Goods \& Services} =  \\ \\
\text{Rest of UK Exports}\|\text{Total Interm. Consumption at Basic Prices}
\end{split} \label{eq:2.5.63}
\end{equation}

\begin{equation} \nonumber
42759 = 42759
\end{equation}\\


\item \textbf {Transfers}

This is the sum of: ``Transfers from RUK'' from the Household Income Account, ``Income from RUK'' from the Corporations Income Account and ``Income from RUK'' from the Government Income Account (15, 27, 41).

\begin{equation}
\begin{split}
\text{Transfers} =  \\ \\
\text{Transfers from RUK}^\text{Households}_\text{IncExp}\\
+\text{Income from RUK}^\text{Corporations}_\text{IncExp}\\
+\text{Income from RUK}^\text{Government}_\text{IncExp}
\end{split} \label{eq:2.5.64}
\end{equation}

\begin{equation} \nonumber
27857 = 1852+5964+20041
\end{equation}\\


\item \textbf {ROW Expenditure in Scotland}

Totals Figure: This is the sum of the two cells below: Goods \& Services and Transfers.

\begin{equation}
\begin{split}
\text{ROW Expenditure in Scotland} =  \\ \\
\text{Goods \& Services}^\text{External}_\text{IncExp}\\
+\text{Transfers}^\text{External}_\text{IncExp}
\end{split} \label{eq:2.5.65}
\end{equation}

\begin{equation} \nonumber
27378 = 19178+8201
\end{equation}\\


\item \textbf {Goods \& Services}

This is the ``Total intermediate consumption at basic prices'' – ``Rest of world exports''. \cite{ScotGov2013a}

\begin{equation}
\begin{split}
\text{Goods \& Services} =  \\ \\
\text{Rest of World Exports}\|\text{Total Interm. Consumption at Basic Prices}
\end{split} \label{eq:2.5.66}
\end{equation}

\begin{equation} \nonumber
19178 = 19178
\end{equation}\\


\item \textbf {Transfers}

This is the sum of: ``Transfers from ROW'' from the Household Income Account and ``Income from ROW'' from the Corporations Income Account (16 \& 28).

\begin{equation}
\begin{split}
\text{Transfers} =  \\ \\
\text{Transfers from ROW}+\text{Income from ROW}
\end{split} \label{eq:2.5.67}
\end{equation}

\begin{equation} \nonumber
8201 = 2237+5964
\end{equation}\\



\item \textbf {Tourist Expenditure in Scotland}

This is the sum of the ``Non-resident household expenditure in Scotland'' (under ``Final consumption expenditure'') - ``Total intermediate consumption at basic prices'' and ``Taxes less subsidies on products''. \cite{ScotGov2013a}

\begin{equation}
\begin{split}
\text{Tourist Expenditure in Scotland} =  \\ \\
\text{Final Consumption Expenditure Non-Resident}\\
\text{Household Expenditure in Scotland}\\
\|\text{Total Interm. Consumption at Basic Prices}\\
\text{Final Consumption Expenditure Non-Resident}\\
\text{Household Expenditure in Scotland}\\
\|\text{Taxes Less Subsidies on Products}
\end{split} \label{eq:2.5.68}
\end{equation}

\begin{equation} \nonumber
2921 = 2599+322
\end{equation}\\


\item \textbf {Total Expenditure}

This is the sum of the above cells: ``UK expenditure in Scotland'', ``ROW expenditure in Scotland'' and ``Tourist expenditure in Scotland''.

\begin{equation}
\begin{split}
\text{Total Expenditure} =  \\ \\
\text{UK Expenditure in Scotland}^\text{External}_\text{IncExp}\\
+\text{ROW Expenditure in Scotland}^\text{External}_\text{IncExp}\\
+\text{Tourist Expenditure in Scotland}^\text{External}_\text{IncExp}
\end{split} \label{eq:2.5.69}
\end{equation}

\begin{equation} \nonumber
100894 = 70595+27378+2921
\end{equation}\\


\item \textbf {Surplus/Deficit}

This is the balance of the External Accounts’ ``Total income'' minus ``Total expenditure'' (57 - 65).

\begin{equation}
\begin{split}
\text{Surplus/Deficit} =  \\ \\
\text{Total Income}^\text{External}_\text{IncExp}\\
-\text{Total Expenditure}^\text{External}_\text{IncExp}
\end{split} \label{eq:2.5.70}
\end{equation}

\begin{equation} \nonumber
-10086 = 90808-100894
\end{equation}\\



\pagebreak


\begin{center}
\textbf{\LARGE Goods and Services Trade Balance}
\end{center}

\item \textbf {RUK}

This is the balance of the ``Goods and Services'' of ``UK expenditure in Scotland'' minus those of ``UK income from Scotland'' (59 - 51).

\begin{equation}
\begin{split}
\text{Goods \& Services Trade Balance with RUK} =  \\ \\ 
\text{RUK Goods \& Services Expenditure in Scotland}^\text{External}_\text{IncExp}\\
-\text{RUK Goods \& Services Income from Scotland}^\text{External}_\text{IncExp}
\end{split} \label{eq:2.5.71}
\end{equation}

\begin{equation} \nonumber
-12020 = 42739-54759
\end{equation}\\


\item \textbf {ROW}

This is the balance of the ``Goods and Services'' of ``ROW expenditure in Scotland'' minus those of ``ROW income from Scotland'' (62 - 55).

\begin{equation}
\begin{split}
\text{Goods \& Services Trade Balance with ROW} =  \\ \\ 
\text{ROW Goods \& Services Expenditure in Scotland}^\text{External}_\text{IncExp}\\
-\text{ROW Goods \& Services Income from Scotland}^\text{External}_\text{IncExp}
\end{split} \label{eq:2.5.72}
\end{equation}

\begin{equation} \nonumber
181 = 19178-18997
\end{equation}\\



\pagebreak

\item \textbf {RUK}

Taking the ``UK expenditure in Scotland'' from the External Accounts, the ``Tourist expenditure in Scotland'' is added to it. This is then multiplied by the share attributed to UK versus ROW tourist and subsequently subtracted by ``UK income from Scotland'' (58,64,51).  \cite{ONS2010a}\\

\begin{equation}
\begin{split}
\text{Total Balance of Payments RUK} =  \\ \\
\text{RUK Expenditure in Scotland}^\text{External}_\text{IncExp}\\
+(\text{RUK Share of Tourist Expenditure in Scotland}\\
*\text{Tourist  Expenditure in Scotland}^\text{External}_\text{IncExp})\\
-\text{RUK Income from Scotland}^\text{External}_\text{IncExp}
\end{split} \label{eq:2.5.73}
\end{equation}

\begin{equation} \nonumber
5215 = 70595+(0.6*2921)-67133
\end{equation}\\


\item \textbf {ROW}

Taking the ``ROW expenditure in Scotland'' from the External Accounts, the ``Tourist expenditure in Scotland'' is added to it. This is then multiplied by the share attributed to ROW versus UK tourist and subsequently subtracted by ``ROW income from Scotland'' (61,64,54). \cite{ONS2010a}

\begin{equation}
\begin{split}
\text{Total Balance of Payments ROW} =  \\ \\
\text{ROW Expenditure in Scotland}^\text{External}_\text{IncExp}\\
+(\text{ROW Share of Tourist Expenditure in Scotland}\\
*\text{Tourist  Expenditure in Scotland}^\text{External}_\text{IncExp})\\
-\text{ROW Income from Scotland}^\text{External}_\text{IncExp}
\end{split} \label{eq:2.5.74}
\end{equation}

\begin{equation} \nonumber
4871 = 27378+(0.4*2921)-23676
\end{equation}\\


\item \textbf {Total BOP}

Totals Figure: This is the sum of the two cells above (69 \& 70).

\begin{equation}
\begin{split}
\text{Total Balance of Payments} =  \\ \\
\text{RUK Total Balance of Payments}+\text{ROW Total Balance of Payments}
\end{split} \label{eq:2.5.75}
\end{equation}

\begin{equation} \nonumber
10086 = 5215+4871
\end{equation}\\



\pagebreak

\begin{center}
\textbf{\LARGE External Balance}
\end{center}

\item \textbf {RUK Total Flows Balance}

This is the balance of ``UK income from Scotland'' minus ``UK expenditure in Scotland'' (51 - 58).

\begin{equation}
\begin{split}
\text{RUK Total Flows Balance} =  \\ \\
\text{RUK Income from Scotland}- \text{RUK Expenditure in Scotland}
\end{split} \label{eq:2.5.76}
\end{equation}

\begin{equation} \nonumber
-3462 = 67133-70595
\end{equation}\\


\item \textbf {ROW Total Flows Balance}

This is the balance of ``ROW income from Scotland'' minus ``ROW expenditure in Scotland'' (54 - 61).

\begin{equation}
\begin{split}
\text{ROW Total Flows Balance} =  \\ \\
\text{ROW Income from Scotland}- \text{ROW Expenditure in Scotland}
\end{split} \label{eq:2.5.77}
\end{equation}

\begin{equation} \nonumber
-3703 = 23676-27378
\end{equation}\\


\item \textbf {Tourist Balance}

Corresponding figure: ``Tourist expenditure in Scotland'' in the External Accounts (64).\\

\begin{equation}
\begin{split}
\text{Tourist Balance} =  \\ \\
-\text{Tourist Expenditure in Scotland}^\text{External}_\text{IncExp}
\end{split} \label{eq:2.5.78}
\end{equation}

\begin{equation} \nonumber
-2921 = -2921
\end{equation}\\


\item \textbf {RUK/ROW Surplus/(Deficit), Lending/(Borrowing) with Scotland}

Totals Figure: This is the sum of the three cells above (72 to 74).

\begin{equation}
\begin{split}
\text{RUK/ROW Total External Balance} =  \\ \\
\text{RUK Total Flows Balance}^\text{External Balance}_\text{IncExp}\\
+\text{ROW Total Flows Balance}^\text{External Balance}_\text{IncExp}\\
+\text{Tourist Balance}^\text{External Balance}_\text{IncExp}
\end{split} \label{eq:2.5.79}
\end{equation}

\begin{equation} \nonumber
-10086 = (-3462)+(-3703)+(-2921)
\end{equation}\\

\pagebreak

\begin{center}
\textbf{\LARGE Shares}
\end{center}

\begin{equation}
\begin{split}
\text{OVA Repatriated to RUK} =  \\ \\
\text{OVA Repatriated}*\text{\%age of UK-owned firms} 
\end{split} \label{eq:2.5.80}
\end{equation}

\begin{equation} \nonumber
3768 = 29456*13\%
\end{equation}\\


\begin{equation}
\begin{split}
\text{OVA Repatriated to ROW} =  \\ \\
\text{OVA Repatriated}*\text{\%age of ROW-owned firms} 
\end{split} \label{eq:2.5.81}
\end{equation}

\begin{equation} \nonumber
4560 = 29456*15\%
\end{equation}\\


\begin{equation}
\begin{split}
\text{Scottish Share of Total UK OVA} =  \\ \\
\text{Scottish OVA}\div \text{UK OVA} 
\end{split} \label{eq:2.5.82}
\end{equation}

\begin{equation} \nonumber
8.31\% = 38441/462590
\end{equation}\\


\begin{equation}
\begin{split}
\text{Scottish Share of Total Household OVA} =  \\ \\
\text{Scottish Household OVA}\\ 
\div (\text{Scottish Household OVA}+ \text{Scottish Corporate OVA}) 
\end{split} \label{eq:2.5.83}
\end{equation}

\begin{equation} \nonumber
15\% = 5289/(5289+29456)
\end{equation}\\


\begin{equation}
\begin{split}
\text{Scottish Share of Total Corporate OVA} =  \\ \\
\text{Scottish Corporate OVA}\\ 
\div (\text{Scottish Household OVA}+ \text{Scottish Corporate OVA}) 
\end{split} \label{eq:2.5.84}
\end{equation}

\begin{equation} \nonumber
85\% = 29456/(5289+29456)
\end{equation}\\


\begin{equation}
\begin{split}
\text{Scottish GDP Share} =  \\ \\
\text{Scottish GDP at market prices}\\ 
\div \text{UK GDP at market prices} 
\end{split} \label{eq:2.5.85}
\end{equation}

\begin{equation} \nonumber
8.22\% = 115167/1401863
\end{equation}\\


\begin{equation}
\begin{split}
\text{Scottish Population Share} =  \\ \\
\text{Population Estimate Scotland}\\ 
\div \text{Population Estimate UK} 
\end{split} \label{eq:2.5.86}
\end{equation}

\begin{equation} \nonumber
8.41\% = 5194/61792
\end{equation}\\


\begin{equation}
\begin{split}
\text{Scottish Household Share} =  \\ \\
(\text{Households and Dwellings Estimate for Scotland}_\text{2001}*3/11)\\
+(\text{Households and Dwellings Estimate for Scotland}_\text{2011}*8/11)\\ 
\div (\text{Households Estimate for the UK}_\text{2001}*3/11)\\
+(\text{Households Estimate for the UK}_\text{2011}*8/11)\\ 
\end{split} \label{eq:2.5.87}
\end{equation}

\begin{equation} \nonumber
9\% = (2.2*3/11 + 2.4*8/11) \div (24.5*3/11 + 26.3*8/11)
\end{equation}\\


\end{enumerate}




% ------------------------------------------------------------------------
